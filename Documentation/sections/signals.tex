\section{Semnalele EEG}

\quad Semnalele electroencefalografice (EEG) sunt o serie de monitorizari
temporale asupra activitatii electrice a creierului uman. Activitatea ce este 
urmarita consista din fluctuatiile campului electromagnetic produs de activitatea
sincronizata a populatiilor de neuroni din diverse sectiuni ale creierului. 
Inregistrarea si captarea semnalelor EEG este utila in analiza electroencefalografica
deoarece stimuleaza abilitatea de a vizualiza si surprinde in timp schimbarile apraute
atunci cand un subiect uman dezvolta procese cognitive.

%\leavevmode \\


\newpage
\quad Pentru a putea capta semnalele EEG, se monteaza pe capul subiectului o casca
ce contine o serie de dispozitive sensibile la fluctuatiile campului 
electric, numite electrozi. Subiectul este apoi instruit sa intretina anumite
activitati cognitive, timp in care se inregistreaza activitatea neuronala. 
Electrozii sunt dispusi in asa fel incat sa fie acoperita o arie cat mai mare
a creierului si sunt de obicei grupati astfel incat un grup de eletrozi sa 
inregistreze activitatea unui lob cerebral. Fiindca semnalele captate au valori
mici, de ordinul microvoltilor ($ \mu V $) acestea sunt amplificate, mai apoi 
digitalizate si filtrate. Fiecare electrod corespunde unui canal electroencefalografic.

\begin{figure}[H]
	\centering
	\begin{tabular}{cc}
		\subfloat[Subiectul uman in timpul inregistrarii EEG]{
			\includegraphics[width = 8cm]{casca_eeg.png}} &
		
		\subfloat[Amplasamentul electrozilor]   {
			\includegraphics[width = 6cm]{electrozi.png}}
	\end{tabular}
	%\caption{Descriptorii LBP}
\end{figure}

\quad Din punct de vedere spectral putem considera semnalele EEG ca fiind compuse din
5 game/benzi de frecventa: Delta (1-4 Hz), Theta (4-7 Hz), Alpha (7-12 Hz), Beta(12-30 Hz)
si Gamma (30-100 Hz).

\begin{figure}[H]
	\includegraphics[width=10.5cm]{eeg_bands.png}
	\centering
	\caption{Benzile semnalelor EEG}
\end{figure}

