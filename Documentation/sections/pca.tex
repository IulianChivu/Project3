\section{Analiza Componentelor Principale}

\quad \textcolor{red}{	
Deoarece datele de intrare sunt semnificative ca dimensiune, iar pentru aceasta
lucrare am dispus de resurse relativ modeste de calcul, am luat decizia de a 
selecta doar canalele EEG pe care le-am considerat importante si mai apoi
de a aplica o analiza a componentelor principale . }

Canalele EEG pe care le-am considerat cele mai importante si de care am dispus in 
setul de date sunt urmatoarele: C5, C3, C1, CZ, C2, C4, C6, CP5, CP3, CP1, CPZ,
CP2, CP4, CP6, P7, P5, P3, PZ, P2, P4, P6, P8, FC5, FC3, FC1, FCZ, FC2, FC4, FC6,
F7, F5, F3, F1, FZ, F2, F4, F6, F8, AF3, AF4.
In urma acestei selectii, din cele 64 de canale initiale vor ramane 40, ceea ce 
ne modifica structura datelor de la (993 x 256,000) la (993 x 160,000).

Analiza componentelor principale (PCA) este o metoda prin care se poate reduce
dimensionalitatea datelor de intrare pastrand in acelasi timp cea mai mare
cantitate din informatie. Algoritmul PCA reprezinta o modalitate
de implementare a transformatei Karhunen–Loève. Pentru a reduce dimensionalitatea
datelor dar si de a pastra cat mai multa informatie este necesara o transformare
liniara cu proprietati de unitaritate, conservare si compactare a energiei.
Transformata Karhunen–Loève ne ofera aceste proprietati. Avand la dispozitie
un vector $ \xi = [ \;\; \xi(0) \;\; \xi(1) \;\; ... \;\; \xi(N-1) \;\; ] ^\top \in 
{\rm I\!R}_{(1 \;X \;n)} $
reprezentand datele ce vor fi transformate, putem gasi reprezentarea vectorului in
spatiul transformat ca fiind $ \eta=L\xi $, unde $ L $ este o transformare liniara.
Pentru a determina transformarea ne vom intoarce in spatiul original: $ \tilde{\xi} = L^T \tilde{\eta} $, transformarea inversa este transpusa lui $ L $ datorita proprietatii
de unitaritate iar $ \tilde{\eta} $ este o aproximatie a lui $ \eta $ deoarece ne 
dorim ca in urma transformarii sa reducem din componentele vectorului $ \eta $.
In acest mod, putem determina eroarea de transformare in sensul erorii patratice medii
ca fiind $ \varepsilon = \overline{\left| \left| \xi - \widetilde{\xi} \right| \right| ^ 2} $
Putem gasi transformarea $ L $ daca minimizam expresia erorii astfel incat
$ \epsilon = \epsilon_{min} $. Se poate demonstra matematic ca transformarea $ L $
este formata din vectorii propri ai matricii de covarianta a vectorului $ \xi $
asezati pe coloana si ordonati descrescator dupa valorile proprii corespunzatoare.
\\
Transformata Karhunen–Loève, gaseste o baza vetoriala de reprezentare astfel incat 
datele sa se grupeze de-a lungul a cat mai multe axe pe care varianta este maxima.
Pe acele axe pe care varianta nu este maxima, putem inlocui valorile din setul de 
date cu media axei respective, iar in cazul in care media a fost extrasa inaintea
transformarii, devenind astfel zero, putem eliminia componentele respective.

In cazul nostru, vectorul $ \xi $ este un vector ale carui elemente sunt la
randul lor vectori n-dimensionali, deci vom aplica algoritmul PCA asupra unei matrici,
a carei forma este determinata de setul de date.
\textcolor{red}{Deoarece dispunem de sisteme de calcul cu capacitate de stocare 
relativ modesta, am luat decizia de a media cele 4000 de esantioane ale fiecarui
canal EEG pe ferestre de cate 10 esantioane fara suprapunere.} In urma medierii,
structura setului de date se modifica de la (993 x 160,000) la (993 x 16,000).


\textcolor{red}{Dupa multiple incercari esuate, am ajuns la concluzia ca setul de date
este inca mult prea mare pentru a-l putea procesa cu puterea de calcul si capacitatea
de stocare de care dispunem, astfel ca am luat decizia de a clasifica semnalele 
pentru 2 clase in loc de 11 cum prevedea setul de date initial.} Astfel, structura
datelor se modifica de la (993 x 16,000) la (182 x 16,000), dimensiunea de 182
fiind justificata de numarul de observatii insumat pentru cele doua clase.

\vspace{0.3cm}
In urma aplicarii algoritmului PCA asupra setului de date cu structura de 
(182 x 16,000) am putut reduce numarul de componente de la 16,000 la
137. Structura finala a setului de date ce vor reprezenta datele de intrare pentru 
clasificator va avea forma (182 x 137).
