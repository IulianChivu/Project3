\section{Setul de date}

\quad Pentru aceasta lucrare ne-am folosit de setul de date pus la dispozitie de 
Departametul de Stiinta Calculatoarelor a Universitatii din Toronto: 
\href{http://www.cs.toronto.edu/~complingweb/data/karaOne/karaOne.html}
{The KARA ONE Database}. Acest set de date a fost achizitionat in urma unui studiu
bazat pe aceleasi principii ca cele descrise in sectiunea semnalelor EEG. 
O casca speciala a fost plasata pe capul subiectilor, ce sacneaza 64 de canale
si are electrozii plasati dupa sistemul 10-20; toate datele au fost amplificate
folosind un amplificator SynAmps RT si esantionate la o frecventa de 
esantionare $ F_S  $ = 1 KHz. Timp de 30 pana la 40 de minute subiectii si-au imaginat
rostirea dar au si rostit 7 foneme si 4 cuvinte.


\quad In aceasta lucrare am preluat semnalele EEG filtrate ce au rezultat in urma 
studiului. Au rezultat astfel un numar de 993 de observatii si 64 de canale pentru fiecare
observatie. Timpul alocat pentru o observatie a fost de 5 secunde. Stiind frecventa de 
esantionare de 1 KHz ne rezulta aproximativ 5000 de esantioane pentru fiecare canal.

\begin{figure}[H]
	\includegraphics[width=10cm]{semnale_eeg.png}
	\centering
	\caption{O parte din canalele EEG ale unei observatii}
\end{figure}

\begin{figure}[H]
	\includegraphics[width=12cm]{semnal_eeg_solo.png}
	\centering
	\caption{Canalul 'CB2' al unei observatii intins pe durata de aproximativ 5 secunde}
\end{figure}



\quad Avand la dispozitie toate datele necesare, putem construi structura pe care o vor
avea datele de intrare ale clasificatorului. Numarul de observatii este de 993, iar 
numarul de canale EEG pentru fiecare observatie este de 64, avand in vedere ca la
intrarea clasificatorlui vom introduce cate o observatie, vom considera toate cele 64 de 
canale ca trasaturi, canalele fiind aici reprezentate de esantioanele lor. 
Un canal contine aproximativ 5000 de esantioane, insa acest numar nu este constant pe intreg setul de date. 
Considerand faptul ca, in timpul studiului subiectul a desfasurat aceeasi 
activitate cognitiva de mai multe ori in decursul celor 5 secunde ale unei observatii, am luat decizia de a
prelua esantioanele de la indicele 499 pana la 4499, deci un numar total de 4000
de esantioane.

Cele 64 de canale, acum a cate 4000 de esantioane fiecare, vor fi liniarizate. In 
acest caz datele de intrare vor avea forma (993 x 4000 $ \cdot $ 64)
 =  (993 x 256,000), adica 993 de observatii a cate 256,000 de trasaturi.
